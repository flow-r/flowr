\HeaderA{flow}{Flow constructor}{flow}
%
\begin{Description}\relax
Flow constructor
\end{Description}
%
\begin{Usage}
\begin{verbatim}
flow(jobs = list(new("job")), name = "newflow", desc = "my_super_flow",
  mode = c("scheduler", "trigger", "R"), flow_base_path = "~/flows",
  trigger_path = "", flow_path = "", status = "")
\end{verbatim}
\end{Usage}
%
\begin{Arguments}
\begin{ldescription}
\item[\code{jobs}] \code{list} A list of jobs to be included in this flow

\item[\code{name}] \code{character} Name of the flow. Defaults to \code{'newname'}
Used in \LinkA{submit\_flow}{submit.Rul.flow} to name the working directories.

\item[\code{desc}] \code{character} Description of the flow
This is used to name folders (when submitting jobs, see \LinkA{submit\_flow}{submit.Rul.flow}).
It is good practice to avoid spaces and other special characters.
An underscore '\_' seems like a good word separator.
Defaults to 'my\_super\_flow'. We usually use this to put sample names of the data.

\item[\code{mode}] \code{character} Mode of submission of the flow.

\item[\code{flow\_base\_path}] The base path of all the flows you would submit.
Defaults to \code{\textasciitilde{}/flows}. Best practice to ignore it.

\item[\code{trigger\_path}] \code{character}
Defaults to \code{\textasciitilde{}/flows/trigger}. Best practice to ignore it.

\item[\code{flow\_path}] \code{character}

\item[\code{status}] \code{character} Not used at this time
\end{ldescription}
\end{Arguments}
